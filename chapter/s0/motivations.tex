\section{Motivation}
In this thesis, we mainly focus on following problems about mobile ground robot navigation through sensorimotor learning.
\begin{itemize}
\item \textbf{Mapless Navigation}:
Traditional navigation pipelines are usaslly based on a pre-defiend map about the enviroment, where high resolution sensors are inevitable for the costmap calculation. For mapless motion planners, robots is expected to navigate a short distance under reactive behaviours without prior knowledge of the environment within a small area \cite{bonin2008visual, guzel2013autonomous}. Various vision sensors \cite{guerrero2005visual, guzel2012behaviour, liu2013visual} are experimented for mapless navigation with relatively expensive computation costs.
2D or 3D ange-sensor-based methods \cite{kamon1999range, ulrich1998vfh+} are also conducted on naive local motion plannings like wall following. Robust motion planners with information from low-dimensional sensors are still challenging.

\item \textbf{Navigation in Pedestrian-rich Environments}:
With more frequently appearance of mobile robots aroung human beings, navigation among pedestrians is one of the most popular topic about mobile robot navigation. Not like the traditional navigation in static environmtents, robot must understand and interact with pedestrians with social manners, as interaction-aware motion planning \cite{pfeiffer2018learning}.
Most of previous approaches ask for a precise localization and velocity information of nearby pedestrians. Except that it is still challeging to collect the real-time velocities of other pedestrians, it also restricts those methods to be only applicable for robots mounted with precise perception sensors like 3D Lidars \cite{pfeiffer2016predicting, Chen17_IROS}.
In addition, the pedestrians state estimation is generally time-consuming through traditional methods.
Developing socially compliant navigation policies directly from raw sensor inputs end-to-end is still of great importance.

\item \textbf{Reality Gap for Deep Reinforcement Learning}:
A simulator Learning-based sensormotor policies, especially through deep reinforcement learning, \hl{motivations of reality gap}.

\item \textbf{Randomness and Uncertainty in Policy Learning}:

\hl{End-to-end visual-based imitation learning has been widely applied in autonomous driving. When deploying the trained visual-based driving policy, a deterministic command is usually directly applied without considering the uncertainty of the input data. Such kind of policies may bring dramatical damage when applied in the real world.}

\end{itemize}
